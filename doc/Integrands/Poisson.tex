% $Id$
%%%%%%%%%%%%%%%%%%%%%%%%%%%%%%%%%%%%%%%%%%%%%%%%%%%%%%%%%%%%%%%%%%%%%%%%%%%%%%%%
\documentclass{article}
\usepackage{eqalign}
\topmargin    -13mm
\oddsidemargin  0mm
\evensidemargin 0mm
\textheight   247mm
\textwidth    160mm
\def\n{{n_{\rm d}}}
\def\R{{\rm I\mkern-4muR}}
\def\deriv#1#2{\frac{\partial#1}{\partial#2}}
\def\dV{\:{\rm d}V}
\def\dA{\:{\rm d}A}
%%%%%%%%%%%%%%%%%%%%%%%%%%%%%%%%%%%%%%%%%%%%%%%%%%%%%%%%%%%%%%%%%%%%%%%%%%%%%%%%
\title{The Poisson problem}
\author{\sl IFEM development team}
\begin{document}
\maketitle
%%%%%%%%%%%%%%%%%%%%%%%%%%%%%%%%%%%%%%%%%%%%%%%%%%%%%%%%%%%%%%%%%%%%%%%%%%%%%%%%

\section{Introduction}

This document describes the strong- and associated weak form of the Poisson
problem, which forms the theoretical foundation of this {\sl IFEM} application.
Refer to the class {\tt Poisson} of the source code for the actual integrand
implementation.
To better see the correspondence with the implementation, we here employ the
tensor notation assuming Einstein's summation convention, i.e., double indices
automatically implies a sum, e.g., $a_ib_i := \sum_{j=1}^\n a_j b_j$, where
$\n$ denotes the number of spatial dimensions (1, 2 or 3).
Furthermore, partial derivation with respect to a given coordinate direction
is indicated by a subscript comma, i.e., $a_{,i} := \deriv{a}{x_i}$.

\section{Strong form}

Given a heat source function $f(x_i)$ defined over a domain $\Omega\in\R^\n$,
a heat flux function $h(x_i)$ defined over the boundary $\partial\Omega_h$,
and a function $g(x_i)$ defined over the boundary
$\partial\Omega_g=\partial\Omega\setminus\partial\Omega_h$,
find the scalar function $u(x_i)\in{\cal U}(\Omega)$ satisfying
%
\begin{eqnarray}
  \label{eq:strong}
  \left.\eqalign{
    q_{i,i} &= f \cr
    q_i & = -\kappa_{ij} u_{,j}
  }\right\} &\forall& x_i\in\overline{\Omega} \\[2mm]
  \label{eq:neumann}
  q_i n_i = h &\forall& x_i\in\partial\Omega_h \\
  u = g &\forall&x_i\in\partial\Omega_g
\end{eqnarray}
%
where $\kappa_{ij}$ is the conductivity tensor and
$n_i$ is the outward-directed unit normal vector on $\partial\Omega_h$.
If the conductivity tensor equals the identity tensor, $\kappa_{ij}=\delta_{ij}$,
we have $q_i=-u_{,i}$ and Equation~(\ref{eq:strong}) reduces to
%
\begin{equation}
  -u_{,ii} = f \quad\forall\quad x_i\in\overline{\Omega}
\end{equation}
%
The solution space ${\cal U}(\Omega)$ defines the set of all admissible solution
functions on the domain $\Omega$, with the additional constraint $u(x_i)=g(x_i)
\;\forall x_i\in\partial\Omega_g$.

\section{Weak form}

The weak form is obtained by multiplying Equation~(\ref{eq:strong}) by a test
function $v(x_i)\in{\cal V}(x_i)$ and integrating over the domain $\Omega$, viz.
%
\begin{equation}
  -\int_\Omega \kappa_{ij}\,u_{,ji}\,v\dV = \int_\Omega f\,v\dV
\end{equation}
%
where the test space ${\cal V}(x_i)$ is the same as ${\cal U}(x_i)$, except that
their functions $v(x_i)$ have the constraint $v(x_i)=0\;\forall x_i\in\Omega_g$.
By applying the Green's identity (integration by parts), this is transformed to
%
\begin{equation}
  \int_\Omega \kappa_{ij}\,u_{,j}\,v_{,i}\dV -
  \int_{\partial\Omega} \kappa_{ij}\,u_{,j}\,n_i\,v\dA =
  \int_\Omega f\,v\dV
\end{equation}
%
The boundary integral of the second term can be further transformed by using
that $v(x_i)=0\;\forall x_i\in\Omega_g$ and substituting
Equation~(\ref{eq:neumann}), resulting in
%
\begin{equation}
  \int\limits_\Omega \kappa_{ij}\,u_{,j}\,v_{,i}\dV =
  \int\limits_\Omega f\,v\dV -
  \int\limits_{\partial\Omega_h} h\,v\dA
\end{equation}
%
or simply
%
\begin{equation}
  a(u,v) = l(v)
\end{equation}
%
where we introduce the bilinear form $(a,v)$ and the linear functional $l(v)$ as
%
\begin{eqnarray}
  a(u,v) &:=& \int\limits_\Omega \kappa_{ij}\,u_{,j}\,v_{,i}\dV \\
  l(v)   &:=& \int\limits_\Omega f\,v\dV - \int\limits_{\partial\Omega_h}h\,v\dA
\end{eqnarray}

\section{Energy norms}

The computed finite element solution can be assessed by evaluating some norms.
For a given finite element solution $u^h$ we therefore define its energy norm as
%
\begin{equation}
  U^h = \sqrt{a(u^h,u^h)}
\end{equation}
%
and the corresponding external energy is
%
\begin{equation}
  U_{ext}^h = \sqrt{l(u^h)}
\end{equation}
%
The implementation can therefore be verified by always asserting that
$U^h=U_{ext}^h$ for any problem setup.

\end{document}
